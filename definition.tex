% !TeX encoding = UTF-8
% !TeX root = main.tex

% \begin{definition}[Black-Box OPRF combiner]
%     A \emph{Black-Box OPRF Combiner} is a protocol $\pi$ that uses two subroutines $\pi_0,\pi_1$ and that UC-realizes $\funcOPRF$ if at least one of $\pi_0$ or $\pi_1$ UC-realizes $\funcOPRF$.
% \end{definition}

% \paragraph{Discussion}
% \begin{itemize}
%     \item This might be too strong a requirement. 
%     Depending on the combiner, it might be ok if $\pi_0,\pi_1$ do not UC-realize $\funcOPRF$ but something weaker (like e.g. omu and wkcr). But for now, it's a start.
%     \item Also, in the definition --as is-- there is no distinction on \emph{what} breaks. Actually, it would be good to express somehow that a part of one of the schemes might still be fine (because it holds statistically).
%     \item We might also want to look at non-black-box combiners later, e.g., when looking at 2Hash-style OPRFs.
% \end{itemize}

\begin{definition}[OPRF combiner]
    Let $\pi_0,\pi_1$ be OPRF protocols that are not necessarily secure according to \Cref{def:simple-oprf}, i.e., they might lack some or all of the required properties. 
    A \emph{OPRF Combiner} is an OPRF $\Combiner(\pi_0,\pi_1)$ that uses $\pi_0,\pi_1$ in a black-box manner and that 
    is a secure OPRF according to \Cref{def:simple-oprf}.
\end{definition}

