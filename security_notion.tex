% !TeX encoding = UTF-8
% !TeX root = main.tex

\subsection{Warm-up}
For warming up, let's look at the round-optimal case as e.g., with 2HashDH~\cite{EC:JarKraXu18}.

\begin{definition}[\cite{EC:TCRSTW22}]
    \label{def:simple-oprf}
    A round-optimal OPRF for the function $f$ is a tuple $\OPRF=(\Gen,\Blind,\BlindEvaluate,\Finalize)$ such that the following properties hold:
    \begin{enumerate}
        \item \emph{Correctness:} The client learns the output $f(\key,x)$. More precisely,
            \[\Pr\left[f(\key,x)=y : 
            \begin{matrix}
                pp\gets\Gen(1^\secpar)\\
                (a,\st)\gets\Blind(pp,x)\\
                b\gets\BlindEvaluate(pp,\key,a)\\
                y\gets\Finalize(pp,\st,b)
            \end{matrix}
            \right]=1-\negl(\secpar),\]
            where the randomness is taken of the random coins consumed by $\OPRF$
        \item \emph{Pseudo-randomness:}
            The output of the protocol is indistinguishable from a truly random function even to maliciously behaving clients, i.e., 
            \[
               \advantPR\coloneqq \abs*{\Pr\left[\ExpPseudoR^{0}=1\right]-\Pr\left[\ExpPseudoR^{1}=1\right]} 
              \leq\negl(\secpar)
            ,\]
            where $\ExpPseudoR^b$ is defined in \cref{fig:pseudorandomness}.
            \item \emph{Input Privacy:} The server does not learn anything about the client's input.
            More precisely, 
            \[
               \advantInputPrivTwo\coloneqq \abs*{\Pr\left[\ExpInputPrivTwo^{0}=1\right]-\Pr\left[\ExpInputPrivTwo^{1}=1\right]} 
              \leq\negl(\secpar)
            ,\]
            where $\ExpInputPrivTwo^b$ is defined in \cref{fig:input-privTwo}.
\end{enumerate}
\end{definition}
\sebastian{Also think about the straight-forward definition of input blindness, which is:}
    for all PPT $\AdvA$ and inputs $x\in\InputSpace$ there exists a PPT simulator $\Sim$ such that
    \[
    \left\{
        \AdvA(f(\key,x),a):
    \begin{matrix}
        pp\gets\Gen(1^\secpar)\\
        (a,\st)\gets\Blind(pp,x)\\
    \end{matrix}
    \right\}
    \compeq
    \left\{
        \AdvA(f(\key,x),a):
    \begin{matrix}
        pp\gets\Gen(1^\secpar)\\
        a\gets\Sim(pp,f(\key,x))\\
    \end{matrix}
    \right\}.\]


\begin{figure}
    \fbox{
    \begin{tabular}{lll}
        \begin{tabular}{l}  
            \textbf{Experiment $\ExpPseudoR^b$}\\
            \midrule
            $\randf\getsr\{g:\InputSpace\to\OutputSpace\}$\\
            $pp\gets\Gen()$\\
            $\key\getsr\KeySpace$\\
            $\st_\Sim\gets \Sim.\Init(pp)$\\
            $b'\gets\AdvA^{\Eval(\cdot),\BlindEvaluate(\cdot),\Prim(\cdot)}(pp)$\\
            output $b'$
        \end{tabular}&
        \begin{tabular}{l} 
            \textbf{Oracle $\Eval(x)$}\\
            \midrule
            $y_0\coloneqq f(\key,x)$\\
            $y_1\coloneqq \randf(x)$\\
            return $y_b$
        \end{tabular}\\
        \begin{tabular}{l}
            \textbf{Oracle $\LimEval(x)$}\\
            \midrule
            $q_s \gets q_s+ 1$\\
            if $q_s \leq q$: return $\Eval(x)$\\
            else return $\bot$
        \end{tabular}
  &
        \begin{tabular}{l}
            \textbf{Oracle $\BlindEvaluate(\alpha)$}\\
            \midrule
            $q \gets q+ 1$\\
            $\beta_0\gets\BlindEvaluate(pp,\key,\alpha)$\\
            $(\beta_1,\st_\Sim')\gets\Sim.\BlindEvaluate^{\LimEval(\cdot)}(\alpha,\st_\Sim)$\\
            return $\beta_b$
        \end{tabular}\\
        \begin{tabular}{l}
            \textbf{Oracle $\Prim(x)$}\\
            $y_0\coloneqq\RO(x)$\\
            $(y_1,\st_\Sim')\coloneqq\Sim.\Prim^{\LimEval(\cdot)}(x,\st_\Sim)$\\
            return $y_b$
        \end{tabular}
    \end{tabular}
    }
    \caption{The pseudo-randomness experiment as in \cite{EC:TCRSTW22}, where $f$ denotes the function that the $\OPRF$ protocol is computing.}
    \label{fig:pseudorandomness}
\end{figure}

\begin{figure}
    \fbox{
    \begin{tabular}{ll}
        \begin{tabular}{l}  
            \textbf{Experiment $\ExpInputPrivOne^b$}\\
            \midrule
            $pp\gets\Gen()$\\
            $\key\getsr\KeySpace$\\
            % $i \coloneqq 0$\\
            $b'\gets\AdvA^{\Trans(\cdot,\cdot),\RO(\cdot)}(pp,\key)$\\
            output $b'$
        \end{tabular}&
        \begin{tabular}{l} 
            \textbf{Oracle $\Trans(x_0,x_1)$}\\
            \midrule
            $(\alpha_{0},\st_{0})\gets\Blind(pp,x_0)$\\
            $(\alpha_{1},st_{1})\gets\Blind(pp,x_1)$\\
            $\beta_0\gets\BlindEvaluate(pp,\key,\alpha_0)$\\
            $\beta_1\gets\BlindEvaluate(pp,\key,\alpha_1)$\\
            $y_0\gets\Finalize(pp,\st_{0},\beta_0)$\\
            $y_1\gets\Finalize(pp,\st_{1},\beta_1)$\\
            $\tau_b \coloneqq (\alpha_{b},\beta_b,y_b)$\\
            $\tau_{1-b} \coloneqq(\alpha_{1-b},\beta_{1-b},y_{1-b})$\\
            return $(\tau_b,\tau_{1-b})$
        \end{tabular} \\ 
    \end{tabular}
    }
    \caption{The honest-but-curious-server input-privacy experiment as in \cite{EC:TCRSTW22}.}
    \label{fig:input-privOne}
\end{figure}

\begin{figure}
    \fbox{
    \begin{tabular}{ll}
        \begin{tabular}{l}  
            \textbf{Experiment $\ExpInputPrivTwo^b$}\\
            \midrule
            $pp\gets\Gen()$\\
            $i \coloneqq 0$\\
            $b'\gets\AdvA^{\Req(\cdot,\cdot),\Fin(\cdot,\cdot,\cdot),\RO(\cdot)}(pp)$\\
            output $b'$
        \end{tabular}&
        \begin{tabular}{l} 
            \textbf{Oracle $\Req(x_0,x_1)$}\\
            \midrule
            $i\coloneqq i+1$\\
            $(\alpha_{i,0},\st_{i,0})\gets\Blind(pp,x_0)$\\
            $(\alpha_{i,1},st_{i,1})\gets\Blind(pp,x_1)$\\
            return $(\alpha_{i,b},\alpha_{i,1-b})$
        \end{tabular} \\ 
        \begin{tabular}{l}
            \textbf{Oracle $\Fin(j,\beta,\beta')$}\\
            \midrule
            if $j > i$ then return $\bot$ \\
            $y_b\gets\Finalize(pp,\st_{j,b},\beta)$\\
            $y_{1-b}\gets\Finalize(pp,\st_{j,1-b},\beta')$\\
            return $(y_b,y_{1-b})$
        \end{tabular}
    \end{tabular}
    }
    \caption{The malicious-server input-privacy experiment as in \cite{EC:TCRSTW22}.\sebastian{Actually, this only makes sense for verifiable OPRFs.}}
    \label{fig:input-privTwo}
\end{figure}
\paragraph{Input Privacy against malicious server.}
\cite{EC:TCRSTW22} define two different flavours of the input blindness experiment.
The first, \sebastian{Cref}, protects against an honest-but-curious adversary, while the second, \sebastian{Cref} protects against a malicious server.

However, the way the second experiment is designed, it is not clear how to transfer this guarantee to the non-verifiable case. One could define the experiment without the verification in the second to last step:
\[\text{If } y_0 = \bot \text{ or } y_1 = \bot \text{ then return } \bot.\]
But then, non-verifiable OPRFs (like 2HashDH without ZKP) allow for a trivial attack (as already mentioned by \cite{EC:TCRSTW22}):
\begin{itemize}
    \item The adversary $\advA$ sends messages $x_0,x_1$ to the $\Req$ oracle and receives $\alpha_0,\alpha_1$ in return.
    \item Like an honest server, $\advA$ chooses $\key\gets\KeySpace$ and sets $\beta_0\gets\BlindEvaluate(pp,\key,\alpha_0)$.
    \item $\advA$ chooses $\key'\getsr\KeySpace$ and $\beta^*\gets\BlindEvaluate(pp,\key',\alpha)$ and gives $(\beta_0,\beta^*)$ to the $\Fin$ oracle.
    \item The $\Fin$ oracle outputs $(y_0,y_1)$. $\advA$ can locally compute the expected OPRF output $y_0^*=F_\key(x_0)$. If $y_0=y_0^*$ then the $\advA$ outputs $b'=0$ and else $b'=1$
\end{itemize}
Note that if the experiment chooses $b=1$ then $\Fin$ will respond with $y_0\gets\Finalize(st_0,\beta^*)$. Thus, $y_0$ is computed under a different server key than $y_0^*$. If the OPRF is pseudorandom, then we have $y_0\neq y_0^*$ with overwhelming probability.

\subsection{Original OPRF Definition}
\label{sec:original-def}
OPRFs were introduced by \cite{TCC:FreIshPinRei05}. Their definition looks as follows (adapted to the round-optimal case):
\paragraph{Correctness and Client Privacy}
Let $\family$ be a pseudo-random function family.

\paragraph{Server Privacy}
Consider the real-world experiment \cref{fig:exp-original-server-priv}.
Intuitively, for all adversaries attacking the client in the real world, there is a simulator that
efficiently simulates the view of this attacker even when the distinguisher is given $n$ additional evaluations of the PRF $f$.
It holds for all $(x_1,\dots,x_n,w)\in\InputSpace^{n+1}$ and for all PPT malicious clients $\advA$ in $\ExpOGPrivReal$ that there exists a PPT simulator $\Sim$ that takes $w$ as input and outputs $(a,b,y)$
such that 
\[ \ExpOGPrivReal(x_1,\dots,x_n) \compeq (\Sim(w),\randf(x_1),\dots,\randf(x_n)),\]
where $\randf$ is a random function $\randf:\InputSpace\to\OutputSpace$, and $\ExpOGPrivReal(x_1,\dots,x_n,w)$ denotes the output of $\ExpOGPrivReal$ on inputs $(x_1,\dots,x_n,w)$.

\begin{figure}
    \fbox{
    \begin{tabular}{ll}
        \begin{tabular}{l}  
            \textbf{Experiment $\ExpOGPrivReal(x_1,\dots,x_n,w)$}\\
            \midrule
            $pp\gets\Gen()$\\
            $\key\getsr\KeySpace$\\
            $(a,\st)\gets\advA(pp,w)$\\
            $b\gets\BlindEvaluate(pp,a,\key)$ \\
            $y\gets\advA(pp,b,\st)$\\
            output $(a,b,y,f(\key,x_1),\dots,f(\key,x_n))$
        \end{tabular} 
    \end{tabular}
    }
    \caption{The honest-but-curious-server input-privacy experiment as in \cite{EC:TCRSTW22}.}
    \label{fig:exp-original-server-priv}
\end{figure}


\subsection{Universally Composable OPRFs}
The gold standard is the UC definition of~\cite{ESP:JKKX16}.
\sebastian{TODO: Add some inutition for the functionality here.}
The full ideal functionality is depicted in \Cref{fig:lateExtractionOPRF}.

Note that \Cref{def:simple-OPRF} does not trivially imply UC security.
Beullens et al. give a general framework for constructing a UC secure OPRF. While pseudo-randomness implies OMU, there might be OPRFs that are secure with respect to \Cref{def:simple-oprf} but that are not WKCR.
Also, there are constructions of UC secure OPRFs that do not fit in the
framework from~\cite{EPRINT:BeuFalHes24}, e.g., 2HashDH.

\sebastian{I'm very sure that \Cref{def:simple-oprf} does not imply
UC security because it does not solve the "correlation" problem from
our paper. But I cannot come up with a separating example :(}


% !TeX encoding = UTF-8
% !TeX root = main.tex

\subsection{Relations between security definitions}
In the following we \sebastian{hopefully} establish relations between 
the security definitions from above.

\sebastian{These are conjectures. I will try to prove them:}
% \begin{lemma}
%     \label{lem:UCcorrectness}
%     Let $\pi$ be an OPRF that UC-realizes $\funcOPRF$. Then $\pi$ has correctness (cf. \Cref{def:simple-oprf}).
% \end{lemma}
% \sebastian{I don't know if correctness is implied by UC security. The ideal functionality has no notion of a key. And also not really of a function that is computed.
% One could define $f(x,\key)$ as the output of an interaction of an honest client and an honest server on input $x$ and the $\key$ is the key that the simulator outputs when the server is compromised. But even then $f$ is a different $f$ in every session. E.g., 2HashDH is not really a function of $x$ and $\key$ but of $x,\key,\sid$.}

\begin{lemma}
    \label{lem:UCinputpriv}
    Let $\pi$ be an OPRF that UC-realizes $\funcOPRF$. Then $\pi$ has input privacy against honest-but-curious servers (cf. \Cref{def:simple-oprf}).
\end{lemma}
\begin{proof}
    The proof strategy is to replace the real protocol transcript in $\ExpInputPrivOne$ by transcripts that are produced by the UC-simulator. If $\advA$ would change its output in the modified game then we can turn $\advA$ into a distinguishing environment for the UC-simulation. In the modified game, the transcripts $\tau_b$ and $\tau_{1-b}$ that $\advA$ sees are distributed identically and thus, they are independent of the bit $b$. Consequently, $\advA$ can guess correctly at most with probability $1/2$.
    
    Let $\Sim$ be the UC-simulator. We modify the game $\ExpInputPrivOne$ to obtain the modified game $\ExpInputPrivOne^\Sim$ as depicted in \cref{fig:input-privSimulated}.

    \begin{figure}
        \fbox{
        \begin{tabular}{l}
            \begin{tabular}{l}  
                \textbf{Experiment $\ExpInputPrivOne^\Sim$}\\
                \midrule
                Give input $(\Compromise,\sid)$ to $\Sim$ on behalf of $\Env$\\
                $\Sim$ responds with $\key\in\KeySpace$\\
                $b'\gets\AdvA^{\Trans(\cdot,\cdot),\RO(\cdot)}(pp,\key)$\\
                output $b'$
            \end{tabular}\\
            \\
            \begin{tabular}{l} 
                \textbf{Oracle $\Trans(x_0,x_1)$}\\
                \midrule
                Give input $(\Eval,\sid,\ssid,\User,\Server)$ and $(\Eval,\sid,\ssid',\User,\Server)$ to $\Sim$ on behalf of $\funcOPRF$\\
                Give input $(\SndrComplete,\sid,\ssid,\Server)$ and $(\SndrComplete,\sid,\ssid',\Server)$ to $\Sim$ on behalf of $\funcOPRF$\\
                $\Sim$ responds with $(\alpha,\beta)$ and $(\alpha',\beta')$\\
                Send permission to deliver all messages to $\Sim$ ob behalf of $\Env$\\
                $\Sim$ responds with  $(\RcvComplete,\sid,\ssid,\Server,i)$ and $(\RcvComplete,\sid,\ssid',\Server,i')$ to $\funcOPRF$\\
                If $i= i'$ and $x_0=x_1$, then choose $y\gets\OutputSpace$ and set $y'\coloneqq y$\\
                Else choose $y'\gets\OutputSpace$ and $y'\getsr\OutputSpace$\\
                $\tau \coloneqq(\alpha,\beta,y)$\\
                $\tau' \coloneqq(\alpha',\beta',y')$\\
                return $(\tau,\tau')$
            \end{tabular} \\ 
        \end{tabular}
        }
        \caption{The modified input-privacy experiment $\ExpInputPrivOne^\Sim$. All $\RO$ queries are also forwarded to $\Sim$. (Like, e.g., \cite{EC:JarKraXu18} but different than \cite{EPRINT:BDFH24}, we do not split the $\RcvComplete$ interface here. This is just syntactic.)}
        \label{fig:input-privSimulated}
    \end{figure}

    First note that if $\advA$'s output distribution would change between $\ExpInputPrivOne$ and $\ExpInputPrivOne^\Sim$ then one can construct an environment $\Env_\advA$ that can distinguish the real execution of the protocol from the ideal execution with simulator $\Sim$ and the ideal functionality $\funcOPRF$, which contradicts the UC-security of $\pi$. 
    
    \begin{itemize}
        \item $\Env_\advA$ sends $(\Compromise,\sid)$ to $\Sim$ and on a response $\key$ it runs $\advA$ as in $\ExpInputPrivOne$.
        \item Whenever $\advA$ makes a call to $\Trans(x_0,x_1)$ then $\Env_\advA$ gives input $(\Eval,\sid,\ssid,\User,\Server)$ and $(\Eval,\sid,\ssid',\User,\Server)$ to $\User$ and input $(\SndrComplete,\sid,\ssid,\Server)$ and $(\SndrComplete,\sid,\ssid',\Server)$ to $\Server$.
        Note that consequently, in the ideal world, $\funcOPRF$ gives the same input to $\Sim$ as in $\ExpInputPrivOne^\Sim$ and $\Sim$ computes $(\alpha,\beta)$ and $(\alpha,\beta)$ as in $\ExpInputPrivOne^\Sim$. 
        In the real world, $(\alpha,\beta)$ and $(\alpha',\beta')$ by $\User$ and $\Server$ executing the protocol code, as in $\ExpInputPrivOne$. 
        \item When $\advA$ makes a $\RO$ query, $\Env_advA$ forwards this to its own $\RO$ (i.e., the RO-functionality.)
        \item $\Env_\advA$ allows the delivery of all messages. 
        In the real world that means that the two honest parties will output $y$ and $y'$ exactly as in $\ExpInputPrivOne$. 
        To ensure that
        the honest parties in the ideal world also produce output, $\Sim$ will send messages $(\RcvComplete,\sid,\ssid,i)$ and $(\RcvComplete,\sid,\ssid',i')$ to $\funcOPRF$. 
        The functionality then samples the output values $y$ and $y'$ from the respective random table: $y\gets T_i[x_0]$ and $y'\gets T_{i'}[x_1]$. 
        In other words, if $i\neq i'$ then $y,y'$ are sampled independently. 
        If $i=i'$ but $x_0\neq x_1$ then the values are still sampled independently. 
        Only if both, $i=i'$ and $x_0=x_1$ both are set to the same value. This is exactly, what $\ExpInputPrivOne^\Sim$ does. 
        \item Finally, $\Env_\advA$ outputs what $\advA$ outputs.
    \end{itemize}
    Overall, in the real world, the view of $\advA$ is distributed as in $\ExpInputPrivOne$ and in the ideal world, the view of $\advA$ is distributed as in $\ExpInputPrivOne^\Sim$.

    Now, in $\ExpInputPrivOne^\Sim$, one can observer that $\tau$ and $\tau'$ are distributed the same.
    $x_0$ and $x_1$ are only used to check if $y$ and $y'$ have to be the same or independent. But in both cases their distribution does not depend on $b$.
    That means, $\advA$ has no advantage for winning this game.
\end{proof}

\begin{lemma}
    \label{lem:UCpseudorand}
    Let $\pi$ be an OPRF that UC-realizes $\funcOPRF$ and that has correctness \sebastian{Depends on the correctness result if we need to demand this separately.}. Then $\pi$ has pseudo-randomness (cf. \Cref{def:simple-oprf}).
\end{lemma}
\begin{proof}
    Assume, by way of contradiction, that there is a PPT adversary $\advA$ that has noticeable advantage in the $\ExpPseudoR$ game. Then, one can construct an environment $\Env_\advA$ that distinguishes the real UC-execution from the ideal execution with simulator $\SimUC$ and the ideal functionality
    $\funcOPRF$. (For this proof, we denote by $\Sim$ the simulator from $\ExpPseudoR$ and with $\SimUC$ the UC-simulator.)


    $\Env_\advA$ internally runs $\advA$.
    \begin{itemize}
        \item When $\advA$ sends a query $x$ to its $\Eval$ oracle, $\Env_\advA$ gives input $(\OfflineEval,sid,\ssid,x)$ to $\Server$. The $\Server$ returns $y$ and $\Env_\advA$ gives this as output of $\Eval(x)$ to $\advA$.
        \item When $\advA$ sends a query $\alpha$ to its $\BlindEvaluate$ oracle, then $\Env_\advA$ gives input $(\SndrComplete,\sid,\ssid)$ to $\Server$ and instructs the dummy adversary to give the input $\alpha$ to $\Server$ on behalf of user $\User$. The dummy adversary will report in response a message $\beta$ that was sent by $\Server$ to $\User$. Return this $\beta$ as output of the oracle.
        \item When $\advA$ sends a query $x$ to its $\Prim$ oracle, then $\Env_\advA$ forwards it to its own $\RO$, i.e., the ideal RO-functionality.
    \end{itemize}
    In the end, $\Env_\advA$ outputs what $\advA$ outputs.

    In the following, we argue that if $\Env_\advA$ is in the real-world execution of the UC-experiment, then $\Env_\advA$ perfectly simulates $\advA$'s view in $\ExpPseudoR^0$ and if $\Env_\advA$ is in the ideal-world execution of the UC-experiment, then $\Env_\advA$ perfectly simulates $\advA$'s view in $\ExpPseudoR^1$.

    \paragraph{$\Eval$ queries.} In the real world, when queried with $(\OfflineEval,\sid,\ssid,x)$ the server outputs $f(\key,x)$, where $f$ is the function that is computed by $\pi$ because $\pi$ has correctness. 
    \sebastian{Does $\pi$ really have to output this? Could it output for $\OfflineEval$ a different function than the one computed by the protocol?}
    In the ideal world, the output of $\Server$ is generated by $\funcOPRF$. The functionality sets the value to $T_\Server[x]$, where $T_\Server$ is the random table kept by $\funcOPRF$ for the honest server.
\end{proof}

\begin{corollary}
    \label{coro:UCfromGames}
   Let $\pi$ be an OPRF that UC-realizes $\funcOPRF$ and that additionally has correctness. Then $\pi$ is a secure OPRF according to \Cref{def:simple-oprf}. 
\end{corollary}
This is a direct consequence of \Cref{lem:UCpseudorand}, and \Cref{lem:UCinputpriv}.

\begin{lemma}
    \label{lem:GameInputPrivFromSim}
    Let $\pi$ have simulation-based input-privacy. Then $\pi$ has input-privacy against honest-but-curious servers.
\end{lemma}
\begin{proof}
    \sebastian{TODO}
\end{proof}

\begin{lemma}
    \label{lem:UCFLateExtractFromGames}
    Let $\pi$ be an OPRF satisfying correctness, pseudo-randomness,simulation-based input-privacy and weak key-collision resistance. Then $\pi$ UC-realizes $\FlateExtract$.
\end{lemma}
\begin{proof}
    \sebastian{TODO}
\end{proof}